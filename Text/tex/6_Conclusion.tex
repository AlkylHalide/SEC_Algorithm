\subsection{6. Conclusion}\label{conclusion}

At the start of this thesis the research question was clearly defined.
The theoretical advances had been made, now it only needed the practical
proof of concept. We implemented a working version of the
Self-Stabilizing End-to-End Communication in Bounded Capacity, Omitting,
Duplicating and non-FIFO Dynamic Networks algorithm in TinyOS, according
to the demands of the original paper in which it described the necessary
steps to be taken for the practical version to be completely working.

We completed the practical demands that were needed in order to fulfill
the self-stabilization criteria. We created several versions of the
algorithm, ran the simulations to define the optimization variables, and
we met the demands of the self-stabilization criteria:

\begin{enumerate}
\def\labelenumi{\arabic{enumi}.}
\itemsep1pt\parskip0pt\parsep0pt
\item
  Insert
\item
  Duplicate
\item
  Reorder
\end{enumerate}

Using LibReplay, we tested the algorithms and saw that they successfully
passed the test.

The issues at hand concerning the development of such an algorithm in
TinyOS are still relevant though. The debugging difficulties concerning
sensor networks and distributed systems make it difficult to at times
find a problem in the code. The TinyOS documentation lack severely, and
since the abandonment of this operating system is slowly starting to
become clear, it doesn't look like this is going to improve.

In light of the results we established, and the myriad of problems we
faced, I look forward and take a moment to advice the future researcher
on the possibilities that are still present in this work.

\subsubsection{6.1 Future work}\label{future-work}

As mentioned in the beginning of this work; the algorithm was written as
part of a research group called Gulliver who are doing research and
activities concerning sensor networks. The eventual goal is for this
algorithm to be used in a working environment, but for this is not yet
ready. More research is necessary to understand where exactly in can
deliver it's highest value.

Second, there were a few variables where I didn't had the time to invest
a lot of effort into them during this project. One of them is the
Reed-Solomon Error Corrrecting Code that I took from a library online.
After going through the research, Reed-Solomon was en still is the
perfect candidate for this algorithm as an Error Corrrecting Code. The
only problem is the high complexity; it would take a separate proect to
develop an RS ECC algorithm. But taking it from a online library,
because of time constraints, meant I couldn't dive in the code and see
if I could remove some parts or tweak some elements that could maybe
improve the perfomance even more.

Finally, looking back at the many problems I faced during development,
and the painstakingly slow process of making applications in TinyOS, I
would strongly advise any future researcher to switch to more supported
operating systems like ContikiOS. TinyOS has it's place, but it is
losing a race it can't win anymore. I base my statements on the founder
and main developer of TinyOS, Professor Philip Levis of Stanford
University. A few years ago he wrote an extended article about his
experiences during ten years of TinyOS development. Looking back and
looking at the current status of TinyOS, even he sees the logical trend
of research mobing their aim away from TinyOS because of the non-active
development.
