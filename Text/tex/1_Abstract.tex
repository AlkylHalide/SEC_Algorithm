\subsection{Abstract}\label{abstract}

End-to-end communication is an essential part of any communication
network. The main idea that has driven and still drives the further
development of end-to-end communication in networks of every sort today,
is the need for reliable, stable, and fast file transfer in a
distributed network. The file size should also not be of any importance;
we strive to guarantee reliable data transfer regardless of the amount.

Sensor networks have come to grow in size and popularity over the past
decades. The need for a reliable way of collecting the data that these
sensors measure, has grown with it. Sensors are inherently more prone to
interference of any kind. On top of that, they are often used in harsh
environments where the unreliable communication channel holds a risk for
corrupting the data that the sensors collect. That is why we keep
looking for new ways to improve the communication channel, and provide
reliable transfer of the data that these sensors measure, to the
collection point where the data is processed.

At the Chalmers University of Technology in Göteborg, Sweden,
researchers have thought of a new algorithm that could potentially solve
these problems of bad communication in unreliable networks. They provide
a self-stabilizing, end-to-end communication in bounded capacity,
omitting, duplicating, and non-FIFO dynamic networks.

We do our research towards finding any effort or literature that is
working towards the practical implementation of such an algorithm. Next,
we look into the background of distributed networking and wireless
sensor networks. We explain the thought process behind the algorithm and
the theoretical inner workings.

Before we start the practical implementation, we look at the Challenges
ahead and formulate the central research question.

In the methods section, we provide a detailed look into the development
process of the algorithm. We give an overview of the tools needed to
complete this work, and the obstacles that had to be overcome.

The results form the key to this work, providing the reader with the
proof that the algorithm works in a practical implementation, and not
only on paper. We execute a number of simulations in different
scenario's, and we try to identify the optimal value for the variables
involved in the performance of the algorithm. The last part of this
chapter focuses on the hard, solid proof. Using a debugging tool
developed at Chalmers, we test the self-stabilization criteria and see
if they hold up to the theory.

Finally we provide the reader with a short overview of what can be done
with these results, and we propose a direction in what future research
should go.
