Loose notes and random ideas Alternating Bit Protocol Unit Testing
TinyOS Cooja Combination TinyOS and Cooja LibReplay Eclipse: Yeti Github

Gebruikte tools: Sublime Atom (en waarom niet VIM), Cooja, telosb motes,
\ldots{}

Wat is moeilijkheid ontwerpen algoritme TinyOs (distributed networking
algemeen): klassiek debugging, op F7 drukken en lijn per lijn afgaan,
werkt niet. Speciale debugging tools nodig.

Context vermelden: onderzoek paper gesubsidieerd via KARYON project; wat
is het, doel, welk puzzelstukje is dit algoritme en resultaten in het
grotere geheel?

Acknowledge problems that arose during the work: no or little support to
be found for TinyOS. E.g.: I can add this in the part about future work
→ in presentation of current state of TinyOS by Phil Levis (Lessons
Learned from 10 years of TinyOS development), he mentions the general
move away from TinyOS to more accessible and supported OS' such as
Contiki. This leads to the conclusion that future work will have to
convert my code to Contiki-compatible code. ALSO, on tinyos wiki itself
found this:

``Generally, the best place to start is with the authors of the paper
describing it. Even if they don't maintain an implementation, chances
are they are aware of existing versions of it. Emailing tinyos-help
isn't usually effective.''
{[}protocolhelp{]}{[}http://tinyos.stanford.edu/tinyos-wiki/index.php/FAQ\#I\_am\_looking\_for\_code\_for\_protocol\_or\_system\_.22X.22:\emph{how}do\_I\_find\_it.3F{]}

Contiki vs TinyOS:
\href{https://www.millennium.berkeley.edu/pipermail/tinyos-help/2010-November/048751.html}{contikivstinyos}

Central research question.

Thesis is an original contribution to knowledge: You have identified a
worthwhile problem or question which has not been previously answered
You have solved the problem or answered the question

Keep to the point A concise paper or thesis requires keeping the main
points in mind--ONLY include background information, data, discussion
that is relevant to these points

You may develop computer programs, prototypes, or other tools as a means
of proving your points, but remember, the thesis is not about the tool,
it is about the contribution to knowledge

Why using TinyOS and not Contiki for example? Ask Elad why he suggested
TinyOS.

In original paper of algorithm are a few ``lemma's'' described, which
form the proof for some of the statements made concerning the advantages
of the algorithm. In ``expected results'' section we can use these
lemma's to form an idea of the results we're expecting to see out of the
simulations and experiments.

Researchers describe WSNs as

``An exciting emerging domain of deeply networked systems of low-power
wireless motes with a tiny amount of CPU and memory, and large federated
networks for high-resolution sensing of the environment.'' {[}M. Welsh,
D. Malan, B. Duncan, T. Fulford-Jones, S. Moulton, ``Wireless Sensor
Networks for Emergency Medical Care,'' presented at GE Global Research
Conference, Harvard University and Boston University School of Medicine,
Boston, MA, Mar. 8, 2004.{]}µ

Thesis Outline

\subsection{1. Abstract}\label{abstract}

\subsection{2. Introduction}\label{introduction}

\subsection{3. Background and Literature
review}\label{background-and-literature-review}

\subsubsection{3.1 Current state of the
art}\label{current-state-of-the-art}

\subsubsection{3.2 Theory (explaining the theoretical working of the
algorithm according to the paper
given)}\label{theory-explaining-the-theoretical-working-of-the-algorithm-according-to-the-paper-given}

\subsection{4. Problem statement/research
question}\label{problem-statementresearch-question}

\subsubsection{4.1 Assignment}\label{assignment}

\subsubsection{4.2 Goals (\textasciitilde{}targets,
objectives)}\label{goals-targets-objectives}

\begin{itemize}
\itemsep1pt\parskip0pt\parsep0pt
\item
  Implement algorithm, full freedom to make own design
\item
  Benchmark the algorithm by performing simulations and establishing
  performance
\item
  Optimize the algorithm using the data acquired by the simulations
\item
  Provide proof of self-stabilization concept by using LibReplay to
  insert, duplicate and reorder packets during communication
\item
  Write report, the goal is here to make sure other researchers can
  continue my work in case they want to implement the algorithm in a
  bigger application.
\end{itemize}

\subsubsection{4.3 Challenges}\label{challenges}

Debugging TinyOS applications: ``Bug hunting in sensor networks is
challenging: Bugs are often prompted by a particular, complex
concatenation of events. Moreover, dynamic interactions between nodes
and with the environment make it time-consuming to track and reproduce a
bug.'' Although LibReplay (see Methods section) offers the ability to
debug sensor network applications like in sequential programming
environments, it is still not feasible to do this consistently for a lot
of bugs during development. To use LibReplay for even the smallest bugs
would require a lot of time. All this means that I had to search for
other debugging methods during my work, such as responding to the
compiler output and using printf statements.
\href{http://www.cse.chalmers.se/~olafl/papers/2015-02-ewsn-landsiedel-libreplay.pdf}{libreplay}

TinyOS documentation often contradicting because of different versions
used, and the up-to-date information is for a large part spread out over
the internet in the form of GitHub issues and commits, TinyOS-help
mailing list, etc.

Abandoned TinyOS support ``Generally, the best place to start is with
the authors of the paper describing it. Even if they don't maintain an
implementation, chances are they are aware of existing versions of it.
Emailing tinyos-help isn't usually effective.''
{[}protocolhelp{]}{[}http://tinyos.stanford.edu/tinyos-wiki/index.php/FAQ\#I\_am\_looking\_for\_code\_for\_protocol\_or\_system\_.22X.22:\emph{how}do\_I\_find\_it.3F{]}

\subsection{5. Methods (Describing How You Solved the Problem or
Answered the
Question)}\label{methods-describing-how-you-solved-the-problem-or-answered-the-question}

\subsection{6. Results}\label{results}

\subsubsection{6.1 Data}\label{data}

\subsubsection{6.2 Interpretation: analysis of
results}\label{interpretation-analysis-of-results}

\subsection{7. Conclusion}\label{conclusion}

\subsection{8. Future
work/Recommendations}\label{future-workrecommendations}

Switch to Contiki

\subsection{9. Acknowledgements}\label{acknowledgements}

Elad, Olaf Daniel, David, Henning, Robin

KU Leuven Peter Karsmakers, Patrick Colleman, Hilde Lauwereys, Patricia
Van Genechten, Isabelle Moons

\subsection{10. References}\label{references}

\subsection{11. Appendices}\label{appendices}
