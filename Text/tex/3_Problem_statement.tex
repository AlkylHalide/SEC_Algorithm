\subsection{3. Problem statement/research
question}\label{problem-statementresearch-question}

The research question presented here is simple:

\emph{Can the practical implementation of the presented
self-stabilization algorithm guarantee the reliable FIFO message
delivery over bounded non-FIFO and duplicating channel?}

Using the central research question as a guide throughout this study, we
can clearly identify the main goal we have to work towards: gathering
the necessary data from the simulations and experiments on the
implementation, to establish sound and quality proof that indicates the
correct working of the algorithm according to the hypothesis presented
in the original paper.

\subsubsection{3.1 Objectives}\label{objectives}

The research question presented above translates in a set of clear
objectives. These have to be met in order to have sufficient empirical
data for the hypothesis of the original paper to be confirmed as true.

\begin{itemize}
\itemsep1pt\parskip0pt\parsep0pt
\item
  Implement the algorithm in TinyOS. In doing so I have the full freedom
  available to me, to make my own choices in terms of coding strategies.
  The only requirement is that the original algorithm is at least
  implemented as is presented. When different possibilities present
  themselves for me to implement a certain behavior of the algorithm, I
  can make the choice myself. It is therefore my responsibility to make
  sure that the choice I make is well thought out, and doesn't
  compromise on the deliverables that are set out in these goals.\\
\item
  Establish good performance variables and benchmark the algorithm by
  performing simulations in function of the predefined performance
  variables.\\
\item
  Optimize the algorithm by adjusting the performance variables. These
  adjustments should be made based upon the data and results that were
  acquired from the simulations.\\
\item
  Provide the necessary proof of the self-stabilization concept by using
  the LibReplay tool to insert, duplicate and reorder packets during
  communication.\\
\item
  Write the report, keeping in mind that future researchers should be
  able to continue or expand on my work in case they want to implement
  the algorithm in a bigger application or adjust based on new
  information.
\end{itemize}

\subsubsection{3.2 Challenges}\label{challenges}

During every empirical research study, you are bound to come across
challenges which will sometimes require to adjust the planning or find a
way around these challenges. I can't think of a field of study where
this is more true than in Computer Science. The biggest obstacle in this
study is without a doubt TinyOS itself.

\paragraph{3.2.1 Debugging TinyOS
applications\\\\}\label{debugging-tinyos-applications}

\begin{quote}
``Bug hunting in sensor networks is challenging: Bugs are often prompted
by a particular, complex concatenation of events. Moreover, dynamic
interactions between nodes and with the environment make it
time-consuming to track and reproduce a
bug\cite{landsiedel2015libreplay}.''
\end{quote}

Although LibReplay (see Methods section) offers the ability to debug
sensor network applications like in sequential programming environments,
it is still not feasible to do this consistently for a lot of bugs
during development. To use LibReplay for even the smallest bugs would
require a lot of time. All this means that I had to search for other
debugging methods during my work, such as responding to the compiler
output and using printf statements\cite{wiki2010tinyos}.

\paragraph{3.2.2 TinyOS documentation\\\\}\label{tinyos-documentation}

TinyOS was first released in 2000. Initially it gained a lot of momentum
for being a very good low-power embedded OS, especially in research
fields. Unfortunately it has fallen very quiet around the whole TinyOS
ecosystem in the last five years. This makes the online documentation
often contradicting because of different versions being used, and the
up-to-date information is for a large part vastly spread out over the
Internet in the form of GitHub issues and commits, TinyOS-help mailing
list, random articles on blog sites etc.

\paragraph{3.2.3 Abandoned TinyOS
support\\\\}\label{abandoned-tinyos-support}

TinyOS used to offer support through an email address
(tinyos-help@millennium.berkeley.edu). This tinyos-help mailing list is
still available to visit, but it completely unorganized and a total mess
to go through.

To illustrate my point, I'm going to use a quote here that is taken
directly from the offical TinyOS FAQ page. This is part of the wiki
pages on TinyOS and was written by the TinyOS developers themselves.

\begin{quote}
\emph{``Generally, the best place to start is with the authors of the
paper describing it. Even if they don't maintain an implementation,
chances are they are aware of existing versions of it. Emailing
tinyos-help isn't usually effective.''}\cite{tinyfaq}
\end{quote}
