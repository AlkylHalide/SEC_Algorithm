\subsection{1 Introduction}\label{introduction}

End-to-end communication is in many ways the basics of communication
networks. Sending a message from one point, the sender, to another, the
receiver. It is what we call a basic primitive in communication
networks. The information must be able arrive at the receiver, one part
of the whole message at at a time. No omissions, duplications or
reordering of the data are allowed.

But just like two persons, talking to each other, can't understand each
other anymore when there's too much noise around them, the same goes for
communication networks. Because of this, errors tend to unfold in the
network between the communicating entities. It is especially at moments
when the communication network experiences high loads, that the chance
of errors becomes very high, and eventually we can't guarantee stable
communication anymore.

In this category of problems, Wireless Sensor Networks (WSNs) are one of
the technologies that are inherently the most prone to interference. The
deployment of WSNs has been steadily on the rise in recent years,
particularly due to the availability of sensors that keep getting
smarter, cheaper, and more intelligent due to big advances in sensor
technologies. Another reason of this sudden rise has everything to do
with two buzzwords that have been circulating in the media in the last
couple of years: IPv6 and the Internet of Things (IoT). But WSNs and
sensors in general are often deployed in outside environments and very
harsh conditions, like volcanoes\cite{1607983}. It's no surprise then
that these little devices are quite prone to catching noise. Luckily,
their increase in popularity has brought about a field of researchers
that aim to reduce this inherent interference.

The aim of research in this field has been to remove outside
interference out of the equation as much as possible, so we are able to
guarantee reliable and well performing end-to-end communication.
Removing the interference is not always easy to accomplish, since we
often have little control over the outside world. But we can, however,
target the errors in the communication channel that this interference
causes. Thus, error detection and error correction are two field where
research is most focused, in the hopes of minimizing errors as much as
possible. In combination with these technologies, research looks toward
the concept of self-stabilization. This means that an algorithm should
be able to recover from any arbitrary state, after encountering an error
for example.

Chalmers University, and specifically my supervisor, Elad Michael
Schiller, have developed an algorithm that employs these two main
technologies: error correction codes and self-stabilization. The
algorithm they present can be applied to dynamic networks of bounded
capacity that omit, duplicate and reorder packets.

\subsubsection{1.1 Background and related
work}\label{background-and-related-work}

Gulliver\cite{Pahlavan}, is a platform for studying vehicular systems on
a large scale open source test-bed of low cost miniature vehicles that
use wireless communication and are equipped with onboard sensors. It was
developed at Chalmers University of Technology, and my supervisor was
part of the team that presented it. The paper about self-stabilizazing
end-to-end communication that is the building block of this thesis, was
written as part of a series of papers presented as part of the Gulliver
project.

At the time of presenting the algorithm, this was the only algorithm of
its kind. There many attempts in the general direction, but none could
satisfy all the guarantees that this algorithm can\cite{dolev2012self}.
My assignment is to implement the algorithm and make practical proof
regarding the theoretical statements the original paper makes in terms
of guaranteeing stable communication.

After doing thorough background work, we have come to the conclusion
that there are still no efforts or algorithms produced that can
guarantee the same kind of stable communication. At the time of writing
this report, the best efforts concerning this type of application are
still the ones referred to in the original paper, so I will not repeat
them here. There is also, to the best of my knowledge, no practical
implementation of any kind available.

\subsubsection{1.2 My contribution}\label{my-contribution}

This thesis investigates the practical implementation in TinyOS and
proof of the presented self-stabilization algorithm. This thesis
presents the first, to the best of my knowledge, practical
implementation and proof of a self-stabilizing end-to-end algorithm for
reliable FIFO message delivery over bounded non-FIFO and duplicating
channel.
