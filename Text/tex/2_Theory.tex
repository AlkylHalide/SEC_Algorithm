\subsubsection{2. Theory}\label{theory}

There can be no practice without knowledge coming first, so I will use
this chapter to explain the original algorithm, and to explain how I'll use it to
implement the practical version.

I divided the theoretical algorithm in two parts. This is in accordance
with the way the algorithm is build up, and maps nicely on the actual
development cycle I went through when implementing the algorithm.

\begin{enumerate}
\def\labelenumi{\arabic{enumi}.}
\itemsep1pt\parskip0pt\parsep0pt
\item
  We start with the first attempt algorithm. This forms the foundation
  for the other, more advanced versions. Just like in a house, you want
  a strong foundation.
\item
  Next is the packet formation algorithm. This is a relatively small
  part, but it adds an important feature. To reach the fully advanced
  sender/receiver algorithm as described in the paper, we add Error
  Correction Coding.
\end{enumerate}

\paragraph{First attempt algorithm\\\\We start with this first algorithm
before moving on to the advanced versions. It is a self-stabilizing,
large overhead end-to-end communication algorithm for coping with packet
omissions, duplications, and reordering. We have two sides, a sender
algorithm and a receiver
algorithm.}\label{first-attempt-algorithm-we-start-with-this-first-algorithm-before-moving-on-to-the-advanced-versions.-it-is-a-self-stabilizing-large-overhead-end-to-end-communication-algorithm-for-coping-with-packet-omissions-duplications-and-reordering.-we-have-two-sides-a-sender-algorithm-and-a-receiver-algorithm.}

The sender starts by fetching a batch of messages from the application
layer. Each data packet is put in message of the form \textbf{}, where
\emph{dat} represents the data, or the actual message packet. The sender
then sends the packet to the receiver. The two other variables are the
Alternating Index, and a label. The label increments until \emph{} every
time a new message is send. The Alternating Index will only increment in
modulo 3 when a new message batch is fetched. This means that for every
value of the \emph{Ai}, \emph{} labels map on to it.

The receiver gets the message delivered, reads the values and puts it in
a position in a \emph{packet\_set} array. The positions in the array is
not randomly chosen, but depends on the \emph{label} value. The Receiver
then sends an \emph{ACK} (acknowledgement) message back to the sender.
When the sender receives this message, it knows the current message was
delivered at the destination and it puts the incoming acknowledgement in
the \textbf{ACK\_set} array. The sender keeps the Alternating Index the
same, but increments the label value. During this process where the
receiver gets the message, puts it away, and sends an acknowledgement
message, the sender keeps transmitting the same message. It only stops
when the acknowledgement arrives. This is the principle of the
\textbf{Alternating Bit Protocol} or ABP\cite{afek1989self}, and it is
just slightly different from the well know Stop-and-Wait ARQ
protocol\cite{fantacci1986generalised}, which waits with sending new
messages until an acknowledgement arrives.

Once the receiver has received the full batch of messages and the
incoming label reaches \emph{}, the receiver sets it's \textbf{Last
Delivered Alternating Index} value to the incoming \emph{Ai}, and
delivers the messages in \emph{packet\_set} to the application layer on
the receiver side.

At this point both sides reset their arrays and the cycle start again
with new messages.

\paragraph{Packet formation from messages\\\\To add a form of redundancy
to the algorithm, the \textbf{packet\_set()} function is added. It takes
a batch of messages of length \emph{pl} and size \emph{ml} (per
message), regards this batch of messages as a 2D bit-matrix, and
tranposes this whole matrix. Instead of a \textbf{pl x ml} size matrix
we now have a matrix with \textbf{n} amount of rows, where \emph{n
\textgreater{} ml} because of the added redundancy bits from the Error
Correction
Coding.}\label{packet-formation-from-messages-to-add-a-form-of-redundancy-to-the-algorithm-the-packetux5fset-function-is-added.-it-takes-a-batch-of-messages-of-length-pl-and-size-ml-per-message-regards-this-batch-of-messages-as-a-2d-bit-matrix-and-tranposes-this-whole-matrix.-instead-of-a-pl-x-ml-size-matrix-we-now-have-a-matrix-with-n-amount-of-rows-where-n-ml-because-of-the-added-redundancy-bits-from-the-error-correction-coding.}
