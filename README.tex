\subsection{S²E²C Algorithm}\label{sec-algorithm}

\subsubsection{\emph{Self-stabilizing End-to-End Communication in
(Bounded Capacity, Omitting, Duplicating and non-FIFO) Dynamic
Networks}}\label{self-stabilizing-end-to-end-communication-in-bounded-capacity-omitting-duplicating-and-non-fifo-dynamic-networks}

\subsubsection{\emph{A Practical Approach in
TinyOS}}\label{a-practical-approach-in-tinyos}

This is the main repository for my thesis work, done at Chalmers
University of Technology in Göteborg, Sweden. It represents the final
stage of my Master of Engineering degree in Computer Science. My
assignment was to implement the algorithm developed by my supervisor,
Elad Michael Schiller, in TinyOS and test in different circumstances to
determine its practical functionality, advantages, and limitations.

The original paper can be found on the
\href{http://www.chalmers.se/hosted/gulliver-en/documents/publications}{Gulliver
Publications page}.

The direct link to the full text is available online through
\href{http://link.springer.com/chapter/10.1007\%2F978-3-642-33536-5_14}{Springer
Link}.

\begin{center}\rule{3in}{0.4pt}\end{center}

\subsubsection{Structure}\label{structure}

There are four main folders in which a version of the algorithm can be
found. Each new version builds upon the previous one and adds
functionality, as explained below. This makes it easy to quickly set up
simulations and experiments for each version, and allows to quickly
compare code of each version if needed.

The four versions are:

\begin{enumerate}
\def\labelenumi{\arabic{enumi}.}
\itemsep1pt\parskip0pt\parsep0pt
\item
  First-Attempt
\item
  Advanced-No-ECC
\item
  Advanced-ECC
\item
  Advanced-ECC-Multihop
\end{enumerate}

All versions work through point-to-point communication. This means a
Sender sends his messages to one specific Receiver, and the Receiver
acknowledges the messages back to the original Sender.

\paragraph{First-Attempt}\label{first-attempt}

This is the first attempt version of the algorithm as described in the
original paper. The Sender sends each message with an Alternating Index
value and a unique Label for each message. The Receiver acknowledges
each message upon arrival and the Sender will only send the next message
if the current one has been properly acknowledged. Once the Receiver
receives a certain amount of messages, the algorithm delivers these
messages to the Application Layer. The variable CAPACITY determines this
amount of messages in the network, the value of which can be adjusted
according to the needs of the implementation in the algorithm itself.

\paragraph{Advanced-No-ECC}\label{advanced-no-ecc}

These are the full Sender and Receiver algorithms as described in the
paper. On top of the basic first attempt functionality the algorithm
divides the messages in packets and sends them to the Receiver. There is
no implementation of Error Correcting Codes in this version yet. The
point of this is to observe the performance of the algorithm without the
Error Correction Codes, and then compare it with the performance of the
algorithm once the Error Correcting Codes are added. This way it is much
easier to see if the possibly increased performance of the algorithm
thanks to the Error Correcting Codes weigh up against the added overhead
that they bring along.

\paragraph{Advanced-ECC}\label{advanced-ecc}

This version contains the full Sender and Receiver algorithm as
described in the paper. This is the First Attempt algorithm, together
with the Packet Generation functionality, and the Error Correcting
Codes.

\paragraph{Advanced-ECC-Multihop}\label{advanced-ecc-multihop}

In a real-life situation, it is very likely that the Sender and Receiver
nodes are not within radio distance of each other. To fully observe the
performance of the algorithm in such a real-life environment, it is
therefore necessary to add the functionality of Multi-Hop routing
through an appropriate Multi-Hop algorithm. In this case we have chosen
for the IPv6 Routing Protocol for Low Power and Lossy Networks (RPL,
pronounced `Ripple'). As you will be able to see in the code, this
brings along significant changes to the Sending and Receiving algorithm.
The reason is that this protocol uses UDP functionality to send and
receives messages. To do this it uses custom TinyOS UDP functions as
defined in BLIP 2.0 (Berkely Low-Power IP Stack).

\begin{center}\rule{3in}{0.4pt}\end{center}

\subsubsection{Usage}\label{usage}

We work with a separate Sender and Receiver algorithm. The code is
therefore implemented in each version in two folders; \emph{Send} and
\emph{Receive}. Each of these folder contains four files. I've made the
naming conventions consistent for each file. I will show the structure
for the \emph{Send} algorithm, but it is identical to the
\emph{Receiver} algorithm except for the file names.

\paragraph{Makefile}\label{makefile}

You can customize all the Makefile options if you need something
changed. I'll explain the two most useful ones here, for the rest I
refer to the TinyOS documentation and specifically BLIP 2.0. Both of
these options are only available in the Multi-Hop version.

\texttt{CFLAGS += -DRPL\_ROOT\_ADDR=11}

This changes the address of the root node in the RPL network. The number
represents the node id. You can change this to any arbitrary node id in
the network.

\texttt{PFLAGS += -DIN6\_PREFIX=\textbackslash{}"fec0::\textbackslash{}"}

With this you can set the IPv6 prefix used to address the nodes in the
network.

I've implemented the Printf functionality in all the versions of the
algorithm. To use it, simply use the standard \texttt{printf()}
C-function followed by the \texttt{printfflush()} function to write the
information to the node output.

\paragraph{SECSend.h}\label{secsend.h}

There are two AM\_TYPE numbers declared for the messages sent between
Sender and Receiver. This way you can multiplex the radio channel. Again
you can change this to any arbitrary number.

\begin{verbatim}
AM_SECMSG = 5
AM_ACKMSG = 10
\end{verbatim}

Two different messages travel across the network:

\begin{itemize}
\itemsep1pt\parskip0pt\parsep0pt
\item
  SECMsg: the data messages from the Sender to the Receiver
\item
  ACKMsg: the acknowledgement messages from the Receiver to the Sender
\end{itemize}

\begin{verbatim}
typedef nx_struct SECMsg {
  nx_uint16_t ai;
  nx_uint16_t lbl;
  nx_uint16_t dat;
  nx_uint16_t nodeid;
} SECMsg;
\end{verbatim}

SECMsg defines four fields:

\begin{itemize}
\itemsep1pt\parskip0pt\parsep0pt
\item
  ai: the current Alternating Index
\item
  lbl: the label of the message, which is unique for each message in
  relation to the current Alternating Index
\item
  dat: the data it contains. In my algorithm this is simply an
  incrementing counter value.
\item
  nodeid: the nodeid of the Sender
\end{itemize}

\begin{verbatim}
typedef nx_struct ACKMsg {
    nx_uint16_t ldai;
    nx_uint16_t lbl;
    nx_uint16_t nodeid;
} ACKMsg;
\end{verbatim}

ACKMsg defines three fields:

\begin{itemize}
\itemsep1pt\parskip0pt\parsep0pt
\item
  ldai: the Last Delivered Alternating Index value
\item
  lbl: the label of the message, which is the label of the incoming
  message for which the Receiver acknowledges the arrival.
\item
  nodeid: the nodeid of the Receiver
\end{itemize}

\paragraph{SECSendC.nc}\label{secsendc.nc}

This is the \textbf{Configuration} file for the TinyOS application.
Unless you want to add new functionality to the algorithm, you should
not change anything in here.

\paragraph{SECSendP.nc}\label{secsendp.nc}

The \textbf{Component} file includes the actual operational logic of the
algorithm. Here you can adjust three elements.

The \emph{capacity} of the network, as described in the original paper,
is determined using this variable. This comes down to how many messages
the Receiver will collect before delivering them to the Application
Layer.

\texttt{\#define CAPACITY 15}

In the packet generation function, there are two predefined values. This
function looks at the array of messages, which are fetched from the
Application Layer at the Sender side, as a matrix. It then transposes
this matrix to generate the packets that will be send over the network.

ROWS defines the amount of messages that are retrieved from the
Application Layer. Basically this comes down to the length of the array
of messages. Since the original paper specifies that the algorithm
fetches \emph{(CAPACITY + 1)} messages from the Application Layer on the
Sender side, this value is set by adjusting the CAPACITY variable.

\texttt{\#define ROWS (CAPACITY + 1)}

COLUMNS defines the length of each message in the array of fetched
messages, on a bit level. The counter values, which are sent as the data
part of each message, are defined as \texttt{uint16\_t} or unsigned
16-bit integers. The bit-wise length of each message is therefore 16
bits (the data part at least), which motivates my choice to put the
COLUMNS variable at 16.

\texttt{\#define COLUMNS 16}

I've created this variable to make the algorithm easily scale according
to the amount of nodes being used in the network. Since each Sender
sends to one specific Receiver and vice-versa, the amount of
Sender-nodes in the network (which should be equal to the amount of
Receiver nodes) determines the address node id of the Receiver and
Sender mote in each respective algorithm.

\texttt{\#define SENDNODES 3}
