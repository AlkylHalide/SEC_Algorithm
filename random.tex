\subsection{5 METHODS}\label{methods}

\subsubsection{\emph{Describing How You Solved the Problem or Answered
the
Question}}\label{describing-how-you-solved-the-problem-or-answered-the-question}

\subsubsection{5.1. Tools}\label{tools}

As I discussed earlier, the work of my thesis has several goals. One of
them is providing a solid starting point for a continuation of this
research into future work, in case a replication of my experiments is
needed, or in the case that the algorithm needs to be implemented in a
bigger project.

In this part I'll discuss the tools I used to implement and evaluate the
S²E²C algorithm. By giving a description of the used materials, future
research can start from my exact situation. This way I provide all the
information of this project in the form of this thesis. For a full
explanation of the details of each tool, I refer to my list of
references. Instead, I will focus here on the elements of each tool that
I specifically applied in my work.

\paragraph{5.1.1 TinyOS\\\\}\label{tinyos}

TinyOS is a free, open-source, flexible and application-specific
embedded operating system designed for low-power wireless devices,
specifically targeting Wireless Sensor Networks (WSN). It was developed
by the University of Berkeley California and has since grown to be the
TinyOS Alliance. The initial release was in 2000. At the time of writing
the latest release, and the version I worked with to implement the
algorithm, is version 2.1.3 and was released in 2012.

Wireless sensor networks usually consist of a large number of small
nodes, which are designed to operate with strict resource constraints
concerning memory, and power consumption. These limited resources and
low-power characteristics require a smart operating system that can
withstand these challenges and successfully handle high load operations
while maintaining a certain level of quality in execution. TinyOS meets
these challenges.

Often the comparison is made between TinyOS and other methods for
developing applications on embedded systems. An example of one such
method are the popular Arduino microcontroller boards. Arduino is more
lightweight than TinyOS. The reason for this is simple; Arduino only
offers some small scale, easy to build and modular kits, based on C
code, for microcontrollers and sensors. TinyOS on the other hand, is a
fully developed operating system. Consequently, while it is very easy to
get started with Arduino, the learning curve for TinyOS is quite steep.
For advanced and powerful applications, and large WSNs however, this
pays off in TinyOS' advantage. Especially applications with elements
such as multi-hop routing, reliable dissemination and time
synchronization require an operating system that can efficiently handle
these complex operations.
\href{http://tinyos.stanford.edu/tinyos-wiki/index.php/FAQ}{tinyosfaq}
\href{https://en.wikipedia.org/wiki/TinyOS}{tinyoswiki}
\href{http://www.ijcst.com/vol61/1/32-Praveen-Budhwar.pdf}{praveen}

\subparagraph{5.1.1.1 nesC\\\\}\label{nesc}

TinyOS itself and its applications are written in nesC, a programming
language specifically designed for networked embedded systems. It uses a
event-driven programming model, and improves reliability while reducing
resource consumption. It achieves this by having restrictions on the
programming model. In their initial presentation of the language, the
original developers acclaim that

\begin{quote}
\emph{``based on our experience and evaluation of the language it shows
that it is effective at supporting the complex, concurrent programming
style demanded by the new class of deeply networked systems such as
Wireless Sensor Networks.''}
\href{http://www.tinyos.net/papers/nesc.pdf}{nesc}
\end{quote}

The use of this new programming language makes programming in TinyOS
challenging. On one hand, nesC is quite similar to C. This means that
implementing new features or protocols usually isn't hard. The
difficulties arise when trying to incorporate new code into existing
one. The part where nesc and C differ greatly is in the linking model.
Writing software components isn't very difficult, but combining a set of
components into a working application is quite challenging and complex.
Unfortunately this is something that I was forced to do throughout the
implementation process of my thesis. As I will discuss later in this
report, this made my work quite difficult at times.
\href{http://csl.stanford.edu/~pal/pubs/tinyos-programming-1-0.pdf}{tinyosprogramming}

\subparagraph{5.1.1.2 Hardware support\\\\}\label{hardware-support}

TinyOS can function on a number of supported hardware platforms. For my
thesis, the hardware that was used is the TelosB Mote Platform. This is
the platform used for most of the WSN research at Chalmers.

All of my algorithm tests and experiments were performed in the form of
simulations. This means I did not use actual hardware. Instead, the
simulator provided an simulation image of the TelosB mote. By doing this
I was able to replicate real life experiments very accurately, since the
compiler makes sure my application was compiled specifically for TelosB
motes, by providing the \texttt{telosb} command as an argument to the
\texttt{make}compile function.
\href{http://www.willow.co.uk/html/telosb_mote_platform.php}{telosbspecs}

\subparagraph{5.1.1.3 Execution model:
split-phase\\\\}\label{execution-model-split-phase}

TinyOS has no blocking operations, because every long-running operation
is executed split-phase. This offers significant advantages over a
blocking system.

\begin{itemize}
\itemsep1pt\parskip0pt\parsep0pt
\item
  Split-phase calls do not occupy too much stack memory while they are
  executing.
\item
  The system stays responsive at anytime. All of the application threads
  are kept free from being held down by blocking calls.
\item
  Split-phase operations reduce stack utilization.
\end{itemize}

A blocking system on the other hand, uses a different approach. Here,
the you are never sure whether or not a call for a long running
operation will return or not. The program will most definitely block
because it has to wait until the operation is complete and the call
returns. Split-phase systems on the other hand, return the call
immediately. A simple example of the difference between the two is shown
in the figure below.\\\\

\begin{figure}[htbp]
\centering
\includegraphics[scale=0.5]{/home/evert/Documents/Thesis/Resources/IMAGES/splitphase.png}
\caption{A split-phase operation example. While split-phase operating systems require more code to reach the same goal, they are far more robust.}
\end{figure}

\href{http://tinyos.stanford.edu/tinyos-wiki/index.php/Modules_and_the_TinyOS_Execution_Model}{splitphasesource}

\paragraph{5.1.2 TOSSIM\\\\}\label{tossim}

TOSSIM is the standard simulator for TinyOS. The newest version can be
found in the TinyOS source directory when you install it, since TOSSIM
is a standard part of TinyOS. It simulates entire TinyOS applications,
and works by replacing components with simulation implementations.

For my research I have not used TOSSIM. Instead I was instructed, during
the explanation of my assignment, that I would use Cooja. This means I
will not go into further detail on TOSSIM, but further details can be
found on the wiki page
\href{http://tinyos.stanford.edu/tinyos-wiki/index.php/TOSSIM}{tossimwiki}.

\paragraph{5.1.3 Cooja\\\\}\label{cooja}

Cooja is the WSN simulator I used for my research. It comes standard
with a version of Contiki-OS, and has a few advantages over TOSSIM. It
has advanced capabilities, active support, and it can work with or
without a GUI. This gave me more flexibility for simulating the
algorithm. An added advantage is that the GUI allows for a smaller
learning curve, which meant quicker testing and development.

Contiki-OS or in short Contiki, is the self pronounced ``Open Source OS
for the Internet of Things''. In contrast to TinyOS, it is being
commercially developed and supported.
\href{http://www.contiki-os.org/index.html}{contikihomepage}

When it comes down to it, Contiki and TinyOS and different operating
systems with the same goal: to develop applications with a small
footprint, targeting low-power devices as used in WSNs. My application
is developed in TinyOS for TelosB motes, but I'm using the Cooja
simulator from Contiki. This means using a slightly different approach
of simulating applications, but luckily it is not difficult at all. The
following steps need to be taken to simulate code based on TelosB motes
in Cooja.

\begin{enumerate}
\def\labelenumi{\arabic{enumi}.}
\item
  Cooja is included in the Contiki CVS Git. Make sure you clone the most
  recent version of Contiki from Git.

\begin{verbatim}
git clone https://github.com/contiki-os/contiki.git
/destination-directory/
\end{verbatim}
\item
  Compile the TinyOS application, by executing the appropriate make
  command in the project directory. For TelosB motes this means,
  executing the following command.

\begin{verbatim}
cd /path-to-project-directory/
make telosb
\end{verbatim}
\item
  Go to the Contiki installation directory. Once there, change the
  directory to \texttt{tools/cooja/} (these steps can be done in one
  command) and run \texttt{ant}. This will start compiling Cooja. Once
  it is done, it will greet you with an empty simulation environment.

\begin{verbatim}
cd /path-to-contiki-directory/tools/cooja/
ant run
\end{verbatim}
\item
  Select \emph{File} → \emph{New Simulation}. Give the simulation a
  appropriate name and select if you'd like to use a random seed on
  startup. Once everything is filled in, hit ``Create''.
\item
  Select \emph{Motes} → \emph{Add motes} → \emph{Create a new mote type}
  → \emph{Sky Mote}.
\item
  In the ``Contiki process/firmware'' field, browse to the location
  where you've compiled your code using the \texttt{make telosb}
  command. If you have followed these steps so far, the location will be
  in \texttt{path-to-project-directory/build/main.exe}.
\end{enumerate}

When following these steps, you will be able to simulate your code on
the nodes in Cooja. Since Cooja is a part of Contiki, not all features
of Cooja will work in combination with the TelosB code. During my
research I did not encounter a problem because of this issue.

\paragraph{5.1.4 LibReplay\\\\}\label{libreplay}

LibReplay is a debugging tool, developed by Olaf Landsiedel, Elad
Michael Schiller and Salvatore Tomaselli at Chalmers University of
Technology. It is specifically designed to solve the problem that bugs
in sensor networks present: they are very hard to hunt down because of
several reasons.

\begin{itemize}
\itemsep1pt\parskip0pt\parsep0pt
\item
  Sensor networks are inherently based on distributed networking
  principles. This means that they operate in a non-deterministic
  environment.
\item
  As a result of this first problem, bugs are time-consuming to track
  and reproduce;
\item
  Bugs usually arise because of a very particular, complex sequence of
  events.
\end{itemize}

In sequential programming bugs are hunted down by integrated debugging
capabilities that are found in Integrated Development Environments.
These kind of debugging capabilities such as stepping through code,
breakpoints, and watchpoints would significantly ease the debugging
process. However, we are incapable of pausing program execution on a
node because of the distributed and embedded nature of sensor networks.

\begin{figure}[htbp]
\centering
\includegraphics{/home/evert/Documents/Thesis/Resources/IMAGES/libreplay.png}
\caption{The above figure shows the way LibReplay works in two (three)
steps: logging, processing the data, and replaying the exact state by
feeding the algorithm all execution variables and necessary information
to recreate the bug.}
\end{figure}

LibReplay solves these issues by logging function calls to and from the
application or code of interest. Afterwards it possible to replay the
execution step-by-step in a controlled environment such as a full-system
simulator like Cooja.
\href{http://www.cse.chalmers.se/~olafl/papers/2015-02-ewsn-landsiedel-libreplay.pdf}{libreplaypaper}

I will use LibReplay as a means to run experiments that are not possible
in the normal mode of operation of sensor networks. These experiments
include inserting, duplicating and reordering packets on the fly.

\paragraph{5.1.5 Github\\\\}\label{github}

The most used Version Control System (VCS) for distributed revision
control and source code management (SCM) on the internet is Github. It
is based on Git, which is

\begin{quote}
\emph{``a widely-used, free and open source distributed version control
system designed to handle everything from small to very large projects
with speed and
efficiency.''}\href{https://en.wikipedia.org/wiki/GitHub}{githubwiki}
\end{quote}

It provides a flexible and fast Linux client as well as a web-based
graphical interface. I have used it for the whole duration of my work
for a safe place to store the code, and revert back to working code in
case something went wrong. Therefore all the source code for my thesis
will be made available at
\url{https://github.com/AlkylHalide/SEC_Algorithm}.
\href{https://git-scm.com/}{git}

\paragraph{5.1.6 Atom\\\\}\label{atom}

Atom was my preferred code editor for this implementation and subsequent
research. This section may seem irrelevant, but I am mentioning it here
because it has been vital for me in writing the code for the algorithm.
The reason for this is that I searched very long for some kind of
editor/IDE that had support for the nesC language. There is to the best
of my knowledge unfortunately no other tool available that provides this
coding support. At the time of writing, it is the only solution that
supports the nesC programming language natively without requiring any
other plugin or extra package. It is also a out-of-the-box working
product for nesC, without requiring extra plugins or packages.
\href{https://atom.io/}{atomeditor}

\subsubsection{5.2 Implementation of the
algorithm}\label{implementation-of-the-algorithm}

I divided the development of the algorithm in four phases. At the end of
each phase I delivered a new version of the algorithm. I started the
code from scratch, each new version building upon the previous one and
adding functionality. The end goal was to test each version in a
simulation environment, and subsequently compare the performance of each
version compared to the other ones. I worked according to this
structure, so that it was easy to quickly set up simulations and
experiments for each version and compare them.

The name of each version is in accordance with the conventions as
described in the original paper for my assignment
\href{http://link.springer.com/chapter/10.1007\%2F978-3-642-33536-5_14}{ogpaper}.

\begin{itemize}
\itemsep1pt\parskip0pt\parsep0pt
\item
  First Attempt
\item
  Advanced algorithm
\item
  Full algorithm: advanced algorithm with Error Correction Encoding
\item
  Full algorithm with multi-hop routing
\end{itemize}

All versions are based on point-to-point communication. This means a
Sender sends his messages to one specific Receiver, and the Receiver
acknowledges the messages back to the original Sender.

In the following sections I will highlight and explain parts of the code
for each version. These are parts where either there was no immediate
documentation for available, the provided documentation dated back to an
older version of TinyOS, or I experienced many difficulties trying to
implement it because of incompatibility or collision with other
components.

\paragraph{5.2.1 First attempt
algorithm\\\\}\label{first-attempt-algorithm}

This is the first attempt version of the algorithm as described in the
original
paper\href{http://link.springer.com/chapter/10.1007\%2F978-3-642-33536-5_14}{ogpaper}.
The Sender sends each message with an Alternating Index value and a
unique Label for each message. The Receiver acknowledges each message
upon arrival and the Sender will only send the next message if the
current one has been properly acknowledged. Once the Receiver receives a
certain amount of messages, the algorithm delivers these messages to the
Application Layer. The variable \emph{} determines this amount of
messages in the network, the value of which can be adjusted according to
the needs of the implementation in the algorithm itself.

\subparagraph{5.2.1.1 Printf debugging\\\\}\label{printf-debugging}

As I described earlier in the part on LibReplay, the best and solution
for small scale debugging in TinyOS and nesC, is using the
\texttt{printf()} command. The output gets sent to the serial port of
the mote. Implementing this functionality is straightforward, and a
tutorial on using the TinyOS Printf library can be found on the wiki
page
\href{http://tinyos.stanford.edu/tinyos-wiki/index.php/The_TinyOS_printf_Library}{tinyprintf}.

The difficult part however, is reading the info from the serial port of
the mote. As said earlier, I use the full system simulator Cooja to
simulate and test the algorithm. Remember that Cooja is part of Contiki,
and so there is no direct support for connecting to the virtual mote in
the simulator. The solution for this is opening a command-line interface
and using the built-in TinyOS Java serial \emph{PrintfClient}
application. This application is able to take an argument through which
you can specify along which communication channel the client should
connect. In total it requires three steps to set up this communication
channel.

\begin{enumerate}
\def\labelenumi{\arabic{enumi}.}
\itemsep1pt\parskip0pt\parsep0pt
\item
  Open a Cooja simulation and load an image in a mote.
\item
  Right-click on the mote and select \emph{Mote tools} → \emph{Serial
  socket (SERVER)}. The serial server socket plugin will open. Note the
  serial port, you can change this if you want.
\item
  Open a command-line interface and execute the following command. When
  you start the simulation, the \texttt{printf()} output will appear
  here.
\end{enumerate}

\begin{verbatim}
java net.tinyos.tools.PrintfClient -comm network@127.0.0.1:SERIAL_PORT
\end{verbatim}

\subparagraph{5.2.1.2 Header file
\emph{SECSend.h}\\\\}\label{header-file-secsend.h}

I use the application header file to define two very important elements.
These are crucial to the operation of the algorithm.

The first is the \textbf{AM Type} of each message. In this case this is
\texttt{AM\_SECMSG} and \texttt{AM\_ACKMSG}. \emph{AM} stands for Active
Message. In the next section I'll explain the usefulness of this.

\begin{verbatim}
AM_SECMSG = 5,
AM_ACKMSG = 10,
\end{verbatim}

The second very important element in the header file is the definition
of custom message structure. TinyOS possesses a standard message format
for sending messages over the network, namely \emph{message\_t}. This
message format allows for custom structures which you can use to define
your own message fields. This is perfect for the algorithm, since we
have to send message-specific info with each packet along the network
like the alternating index, a label, the data and the node id.

I have defined two message structures according to the requirements of
the algorithm;

\begin{enumerate}
\def\labelenumi{\arabic{enumi}.}
\itemsep1pt\parskip0pt\parsep0pt
\item
  One structure \emph{SECMsg} with four fields: alternating index,
  label, data, node id. It is used to transmit the data messages.
\item
  The second structure, \emph{ACKMsg}, is used for the acknowledgement
  messages from the receiver. It houses only three fields: last
  delivered alternating index, label, and node id.
\end{enumerate}

\begin{verbatim}
typedef nx_struct SECMsg {
  nx_uint16_t ai;
  nx_uint16_t lbl;
  nx_uint16_t dat;
  nx_uint16_t nodeid;
} SECMsg;

typedef nx_struct ACKMsg {
    nx_uint16_t ldai;
    nx_uint16_t lbl;
    nx_uint16_t nodeid;
} ACKMsg;
\end{verbatim}

\subparagraph{5.2.1.3 Component file
\emph{SECSendC.nc}\\\\}\label{component-file-secsendc.nc}

nesC uses two main types of files: module files and configuration files.
Modules consist of the actual application code, while configurations are
used to assemble other components together. The interfaces are
implemented in the module file, while the connection of the interfaces
with the respective components are provided by the component file. This
is called the \textbf{wiring} of interfaces and components. A nesC
application can be uniformly described by a configuration file that
wires together the components.
\href{http://www.tinyos.net/tinyos-1.x/doc/tutorial/lesson1.html}{compsandmods}

I've explained earlier that one of the difficulties of working with
TinyOS and nesC, is the component linking model. The component file,
ending with a \textbf{C} in accordance with the TinyOS naming
conventions, links the components and interfaces together that form the
nesC application at compile time.

Two lines are especially important, and they refer back to the \emph{AM
Types} that I defined in the header file. These are also the only lines
in the component file that are different between the sender and the
receiver application.

\textbf{Sender}

\begin{verbatim}
components new AMSenderC(AM_SECMSG);
components new AMReceiverC(AM_ACKMSG);
\end{verbatim}

\textbf{Receiver}

\begin{verbatim}
components new AMSenderC(AM_ACKMSG);
components new AMReceiverC(AM_SECMSG);
\end{verbatim}

It's not uncommon to have multiple services using the same radio to
communicate messages across the network. Naturally, these services can't
connect all at the same time.To solve this problem, TinyOS provides the
Active Message (AM) layer as a way to multiplex the access to the
communication channel. Using this \textbf{AM type} to initialize the
sender and receiver components, we can easily filter the messages that
arrive at the receiver channel.

Upon inspection you'll notice the sender and receiver components have
reverse \emph{AM} types assigned to their \emph{AMSenderC} and
\emph{AMReceiverC} component. Expanding upon the explanation of these AM
types, this is logical. The sender sends data messages with a
\emph{SECMsg} structure, so the \emph{AMSenderC} component is
initialized with this AM type. Likewise, it receives acknowledgement
messages from the receiver with a \emph{ACKMsg} structure, so it is
initialized with the \emph{AM\_ACKMSG} AM type. The receiver follows the
same logic, but in the reverse direction.

\subparagraph{5.2.1.4 Module file
\emph{SECSendP.nc}\\\\}\label{module-file-secsendp.nc}

The module file, ending with a \textbf{P} again in accordance with the
TinyOS naming conventions, is the module file. This is where we write
the actual logic of the algorithm. I will expand upon the specific logic
I've implemented in the coming sections.

\paragraph{5.2.2 Unit testing\\\\}\label{unit-testing}

\paragraph{5.2.3 Packet formation from
messages\\\\}\label{packet-formation-from-messages}

These are the full Sender and Receiver algorithms as described in the
paper. On top of the basic first attempt functionality the algorithm
divides the messages in packets and sends them to the Receiver. There is
no implementation of Error Correcting Codes in this version yet. The
point of this is to observe the performance of the algorithm without the
Error Correction Codes, and then compare it with the performance of the
algorithm once the Error Correcting Codes are added. This way it is much
easier to see if the possibly increased performance of the algorithm
thanks to the Error Correcting Codes weigh up against the added overhead
that they bring along.

When the messages are fetched from the application layer at the sender,
these messages are divided into packets according to a special
transformation. We regard the array of messages in bit format as a two
dimensional array. We transpose this whole array, and then divide the
result in even packets.

To obtain this result I've defined a function called
\texttt{packet\_set()}. This is the sequence I make the messages go
through: I fetch the messages from the application layer, convert each
message to its appropriate bitwise representation, the result is put in
a two dimensional matrix, transposed, and converted back to integers to
generate the packets. These packets are then passed on to the sending
function.

\begin{figure}[htbp]
\centering
\includegraphics{/home/evert/Documents/Thesis/Resources/IMAGES/transposefunction.png}
\caption{This figure shows the packet\_set() function which is used to
generate packets. It is clear to see that the sender and receiver
functions are each other's opposites.}
\end{figure}

\paragraph{5.2.4 Error correction
codes\\\\}\label{error-correction-codes}

This version contains the full sender and receiver algorithm as
described in the paper. This is the first attempt algorithm, together
with the packet generation functionality, and the error correcting
codes.

When implementing error correction codes, there are a few options to
choose from. The most commonly found codes used in implementations of
sensor networks and more generally low-power and resource-constrained
devices, are Reed-Solomon codes. They offer high performance and work
reasonably well in this sort of communication because they can make up
for errors of longer length than other Error Correcting Codes.
\href{Resources:\%20FULLTEXT01.pdf\%20aka\%20Performance\%20Evaluation\%20of\%20Error\%20Correcting\%20Techniques\%20for\%20OFDM\%20Systems}{eccvgl}

The disadvantage of working with Reed-Solomon is that they're quite
difficult to implement correctly. To save time on my development cycle,
I've used an external library for Reed-Solomon error correction. It
provides me with a C implementation which I added to my own algorithm
\href{http://www.drdobbs.com/testing/error-correction-with-reed-solomon/240157266}{drhobbs}.

\paragraph{5.2.5 Multi-hop routing
algorithm\\\\}\label{multi-hop-routing-algorithm}

In a real-life situation, it is very likely that the Sender and Receiver
nodes are not within radio distance of each other. To fully observe the
performance of the algorithm in such a real-life environment, it is
therefore necessary to add the functionality of Multi-Hop routing
through an appropriate Multi-Hop algorithm. In this case we have chosen
for the IPv6 Routing Protocol for Low Power and Lossy Networks (RPL,
pronounced `Ripple'). Unfortunately this brings along significant
changes in the code to the Sending and Receiving algorithm. The reason
is that this protocol uses UDP functionality to send and receives
messages. To do this it uses custom TinyOS UDP functions as defined in
BLIP 2.0 (Berkely Low-Power IP Stack).

As we can see, the send and receive functions are now UDP-specific. This
also means that we lose the ability to give AM types to outgoing
messages. More importantly, it means we can't check incoming messages
for their AM type. Luckily the 6LowPan implementation in TinyOs provides
an AM field in the frame format, so I switched to incorporate that in
combination with RPL.
\href{http://www.tinyos.net/tinyos-2.x/doc/txt/tep125.txt}{tep125}

\textbf{UDP receive function}

\begin{verbatim}
event void RPLUDP.recvfrom(struct sockaddr_in6 *from, void *payload,
  uint16_t len, struct ip6_metadata *meta){
\end{verbatim}

\textbf{UDP send function}

\begin{verbatim}
call RPLUDP.sendto(&dest, &btrMsg, sizeof(nx_struct SECMsg));
\end{verbatim}

The use of 6LowPan and RPL routing means we switch to IPv6 addressing
for sending to the other nodes. In point-to-point network, this requires
that the address of the receiving node is known at compile town. Again I
found a nice implementation of this feature already. The general prefix
for IPv6 addressing is set in the Makefile on the following line.

\begin{verbatim}
PFLAGS += -DIN6_PREFIX=\"fec0::\"
\end{verbatim}

This means that all the nodes will have an IPv6 address starting with
\texttt{fec0::\textbackslash{}}. The last part of the IPv6 address is
used to refer to the specific node address. In the case of TinyOS, the
last part can be directly filled in with the receiving node's node id.
This makes it easy for us, since we were already addressing nodes this
way.

\begin{verbatim}
memset(&dest, 0, sizeof(struct sockaddr_in6));
dest.sin6_addr.s6_addr16[0] = htons(0xfec0);
dest.sin6_addr.s6_addr[15] = (TOS_NODE_ID + SENDNODES);
\end{verbatim}

\subsubsection{5.3 Evaluation and
preliminaries}\label{evaluation-and-preliminaries}

\begin{enumerate}
\def\labelenumi{\arabic{enumi}.}
\itemsep1pt\parskip0pt\parsep0pt
\item
  The \textbf{performance} is measured through the time it takes to send
  a packet on its way to a receiver. This performance likely degrades as
  the algorithm becomes more complex. The reasoning behind this is that,
  with every added complexity (and new version of the algorithm), the
  overhead or expense to generate and send the packet, becomes larger.
  With every new version, the data gets edited more and subsequently it
  takes longer for the node to process the data and send the packet.
\item
  I measure as well the \textbf{amount of errors} that build up in the
  output of the Receiver node. An error can indicate either lost or
  corrupted packets and while these errors vary in terms of eventual
  impact on the application layer performance, the quality of
  transmission degrades nonetheless.
\end{enumerate}

There are two variables in the algorithm that influence these criteria.
The variable \emph{} as indicated in the original paper, and the amount
of nodes in the network.

\begin{enumerate}
\def\labelenumi{\arabic{enumi}.}
\itemsep1pt\parskip0pt\parsep0pt
\item
  The variable \texttt{CAPACITY} determines the amount of packets the
  Sender transmits to the Receiver, before fetching another batch of
  messages from the application layer. The prediction is that by
  increasing this number, the amount of errors will likely rise, because
  it will take more processing to Send the messages, and
\item
  The variable \texttt{SENDNODES} keeps track of the amount of Sender
  nodes that are used in the network. I've created this variable to make
  the algorithm easily scale according to the amount of nodes being used
  in the network. Since each Sender sends to one specific Receiver and
  vice-versa, the amount of Sender-nodes in the network (which should be
  equal to the amount of Receiver nodes) determines the address node id
  of the Receiver and Sender mote in each respective algorithm. My
  prediction is that the performance will again decrease the more nodes
  are used in the network. It is my belief that this will especially
  hold true for the implementation with the multi-hop routing algorithm,
  because of the complexity of the algorithm and the extra processing
  that's needed to get the packets to their destination.
\end{enumerate}

\subsubsection{5.3 Simulations}\label{simulations}

We want to achieve two things with these simulations. First of all,
we'll test the performance of each algorithm in normal circumstances by
letting the Sender transmit a very large amount of data to the Receiver
node, multiple times in a row. Secondly, we'll run a sequence of
simulations where we independently adjust the two variables we talked
about earlier. First, we'll set \texttt{CAPACITY} to a fixed number.
Then we will run a number of simulations in a row, and each cycle we
increase the amount of nodes in the network, which will also change the
variable \texttt{SENDNODES}. During each simulation, we measure the
above mentioned criteria i.e., the performance and error count.

We'll repeat this procedure for values of \texttt{CAPACITY} from 1 to
16. The goal is to eventually find optimal values for both variables in
which the performance and the error count balance each other out. We
want to find the values that allow us the best performance while keeping
the error count as low as possible.

\paragraph{5.3.1 Single-hop setup\\\\}\label{single-hop-setup}

To test the first three out of four versions of the algorithm we only
need a single-hop setup between a sender and a receiver. The amount of
total nodes in the network doesn't matter, as long as sender and
receiver are in radio transmission range from each other.

\begin{figure}[htbp]
\centering
\includegraphics[scale=0.5]{/home/evert/Documents/Thesis/Resources/IMAGES/singlehop.png}
\caption{The above figure shows the single-hop setup between two nodes in a Cooja simulation. The blue arrows indicate bi-directional traffic.}
\end{figure}

\begin{figure}[htbp]
\centering
\includegraphics[scale=0.5]{/home/evert/Documents/Thesis/Resources/IMAGES/singlehopmultinode.png}
\caption{We add more nodes to the single-hop network. Now the bi-directional communication is clearly visible. Also notice that, because there is no routing yet, the sender and receiver nodes send their messages in all directions.}
\end{figure}

\paragraph{5.3.2 Multi-hop setup\\\\}\label{multi-hop-setup}

The multi-hop setup requires two extra nodes on top of the normal sender
and receiver nodes.

\begin{enumerate}
\def\labelenumi{\arabic{enumi}.}
\itemsep1pt\parskip0pt\parsep0pt
\item
  \textbf{Root node}: in any RPL network, one node needs to be the root
  node. This node `assists' the router in helping the packets arrive at
  their destination.
\item
  \textbf{PPP-Router}: this is a standard application in the TinyOS app
  directory. It provides routing control for multi-hop, IPv6 networks.
\end{enumerate}

{[}Now we have routing; node 1 works as the root node and the PPP-Router
performs it's routing functions to transfer packets through the
network.{]}{[}coojamultihop{]} {[}coojamultihop{]}:
/home/evert/Documents/Thesis/Resources/IMAGES/coojamultihop.png
``Multi-hop operation example''

\paragraph{5.3.3 Automating simulations using test
scripts\\\\}\label{automating-simulations-using-test-scripts}

To completely automate the simulations so they can run in the
background, without supervision, requires the use of \textbf{Contiki
test scripts}. This is a standard plugin in Cooja. Using this test
script, I wrote a number of Linux Shell Scripts to capture the data
being transmitted by the nodes.

I wrote one starting script, \emph{simauto.sh}, and this is comparable
to the key of a car; it starts the car but doesn't do much itself.

\begin{verbatim}
. /home/evert/tinyos-main/apps/SEC/coojasim.sh &
. /home/evert/tinyos-main/apps/SEC/serial_connect.sh &
wait
echo "Processes complete"
\end{verbatim}

In the code above I call two other scripts: \emph{coojasim.sh} and
\emph{serial\_connect.sh}. The first one starts Cooja in \textbf{nogui}
mode (this means Cooja works through the command line, without a
graphical user interface) and it automatically loads the preferred
simulation that you want to run. The second script connects to the
serial interface of the mote(s) in Cooja, which I already discussed in
an earlier paragraph of this methods section.

The trick of the script lies in the parallel processing. Behind every
script is the \emph{\&} symbol, and the second to last line has the
keyword \emph{wait}. This means the script will call both scripts and
let them run in the background. This has to do again with the
non-sequential way that sensor networks work. These scripts need to be
called at the same time, because you can't execute the algorithm and
sequentially listen to the serial port; you won't receive any data.

\subsubsection{5.4 Experiments}\label{experiments}

The experiments using LibReplay have one goal: to provide indefinite
proof of the self-stabilization criterion that is discussed in the
original paper. The criterion for self-stabilization requires that,
starting from any arbitrary state, the system can return to a stable
state in which communication goes smooth again. Practically this results
in the following experiments that need to be done.

\begin{enumerate}
\def\labelenumi{\arabic{enumi}.}
\itemsep1pt\parskip0pt\parsep0pt
\item
  Insert,
\item
  duplicate,
\item
  reorder packets during algorithm execution.
\end{enumerate}

The goal is to show that, despite disturbing the communication by
changing the packet transmission, the receiver will be able to correctly
and swiftly deliver the data to the application layer.

\paragraph{5.4.1 LibReplay setup\\\\}\label{libreplay-setup}

LibReplay works quite simple in itself. Unfortunately TinyOS doesn't
always work without problems with the features provided by LibReplay. As
discussed earlier, LibReplay works in two steps;

\begin{enumerate}
\def\labelenumi{\arabic{enumi}.}
\itemsep1pt\parskip0pt\parsep0pt
\item
  Implement logging components on the interfaces you wish to log.
  Subsequently, LibReplay logs the function calls and current
  environment variables. The algorithm execution state is completely
  logged and kept track off.
\item
  Using the same interfaces we logged in the first step, implement the
  replay components. This way the exact execution state can be fed back
  to TinyOS and we can replicate the exact circumstances in which a
  certain bug presented itself. Implementing the replay components is
  just a matter of replacing the logging components with the replay
  components.
\end{enumerate}

\begin{figure}[htbp]
\centering
\includegraphics{/home/evert/Documents/Thesis/Resources/IMAGES/libreplaylog.png}
\caption{The above figure shows the replay component added to the
receiving interface. For our algorithm we'll log both the sender and
receiver interfaces at both the sender and receiver motes.}
\end{figure}
